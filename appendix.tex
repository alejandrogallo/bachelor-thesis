\newpage
\appendix
\section{Program to look for counter-examples}
\label{appendix:section:algorithm}
The following algorithm has the following structure:
\begin{enumerate}
\item Generate $\alpha, \beta$ and $u$.
\item Determine the optimal value of $t$:
    \begin{enumerate}
    \item if $\alpha \leq t_{1}$, we can choose $t = \alpha$. 
    \item else, i.e., if $\alpha > t_{1}$ then $t = t_{1}$. 
    \end{enumerate} 
\item Calculate $\lambda_{\alpha, \beta} = \lambda(t; \alpha, \beta, u)$. 
\item Repeat step $2$ and $3$ interchanging $\alpha$ by $\beta$ and $\beta$ by $\alpha$. The result will be the $\lambda_{\beta, \alpha} = \lambda(t; \beta, \alpha, u)$. 
\item If $\lambda_{\alpha,\beta} = \lambda_{\beta, \alpha} $, go again to step 1. Else, we have our examples. 
\end{enumerate}
{\small\begin{verbatim}
from math import sqrt, ceil;
#Definition of the formula for the eigenvalues. 
#There is an optional parameter, mode, 
#which one can change to give us only the 
#+-solutions or the maximum of both eigenvalues. 
#By default it is set to max. 
def eig(alpha,beta,t,u,mode = 'max'):
    #calculate the part with and 
    #without the square root of the formula.
    no_sqrt = (alpha - t + 2 - beta)/2;
    yes_sqrt = (sqrt((beta+alpha-t)**2-4*beta*(alpha-t)*u**2))/2;
    #cases considering the different values of the mode parameter.
    if mode == 'plus':
        value = no_sqrt+yes_sqrt;
        return value;
    elif mode == 'minus':
        value = no_sqrt-yes_sqrt;
        return value;
    elif mode == 'max':
        #this will return the norm, i.e. the maximum of the eigenvalues. 
        value = max(no_sqrt+yes_sqrt, no_sqrt-yes_sqrt);
        return value;
    else:
        return False;
def t1(c, u):
    if u==0:
        return 1;
    else:
        return (1-c)/(1-c + c*u**2);
#This is to calculate the constraint, 
#we just compare t1(beta) and alpha
def tconstraint(alpha,beta,u):
    t1ab = t1(beta,u);
    t1ba = t1(alpha,u);
    if alpha <=t1ab:
        constraint_ab = alpha;
    else:
        constraint_ab = t1ab;
    if beta <= t1ba:
        constraint_ba = beta;
    else:
        constraint_ba = t1ba;
    #We return the values in the format of a dictionary
    return {'constraint_ab':constraint_ab,
     'constraint_ba':constraint_ba};
#This function will look for suitable points for
# which the norm varies in dependence of 
#the permutation of alpha and beta. 
#If count is set to true then we will count the 
#numbers that are in the different 
#sets M<<, M<> or M>>
def look_for_points(N,all = False, count = False):
    #we create a grid in (0,1) of N-2 points. 
    #The precision we will use is 5 decimal places
    precision = 5;
    #the number of total points (alpha,beta,u) which are looked
    total_points = (N+1)*(N+1)*N;
    #help variable for the percentage process
    perc_step = 0;
    count_total = 0;
    count_dif =0;
    count_great_great=0;
    count_great_less=0;
    count_less_less=0;
    count_less_great=0;
    eq_count_great_great=0;
    eq_count_great_less=0;
    eq_count_less_less=0;
    eq_count_less_great=0;
    for i in range(0,N+1):
        for j in range(0,N+1):
            #u is in [0,1)
            for k in range(1,N+1):
                alpha = round((1-i/N)*1 + (i/N)*0,precision);
                beta = round((1-j/N)*1 + (j/N)*0,precision);
                u =round((1-k/N)*1 + (k/N)*0,precision);
                constraint = tconstraint(alpha,beta,u);
                tab = round(constraint['constraint_ab'],precision);
                tba = round(constraint['constraint_ba'],precision);
                ab = round(eig(alpha, beta, tab,u),precision);
                ba = round(eig(beta,alpha,tba,u),precision);
                count_total+=1;
                #Percentage process
                if count_total/total_points >= perc_step:
                    print(str(ceil(100*perc_step))+'%');
                    perc_step += .1;
                if ab != ba:
                    count_dif+=1;
                    points = {'alpha':alpha, 'beta':beta, 'u':u,
                     'tab':tab, 'tba':tba, 'ab':ab, 'ba':ba};
                    if all:
                        if alpha > t1(beta,u):
                            if beta > t1(alpha,u):
                                count_great_great+=1;
                            elif beta < t1(alpha,u):
                                count_great_less+=1;
                        elif alpha < t1(beta,u):
                            if beta > t1(alpha,u):
                                count_less_great+=1;
                            elif beta < t1(alpha,u):
                                count_less_less+=1;
                    else:
                        #When we only want a suitable point, 
                        #we get the first suitable point. 
                        return points; 
                else:
                    if all:
                        if alpha > t1(beta,u):
                            if beta > t1(alpha,u):
                                eq_count_great_great+=1;
                            elif beta < t1(alpha,u):
                                eq_count_great_less+=1;
                        elif alpha < t1(beta,u):
                            if beta > t1(alpha,u):
                                eq_count_less_great+=1;
                            elif beta < t1(alpha,u):
                                eq_count_less_less+=1;
    if(count and all):
        print('---------------------');
        print('total points = '+str(count_total));
        print('\ntotal suitable points = '+str(count_dif));
        print('   total great great points = '+str(count_great_great));
        print('   total great less points = '+str(count_great_less));
        print('   total less great points = '+str(count_less_great));
        print('   total less less points = '+str(count_less_less));
        print('\ntotal nonsuitable points = '+str(count_total-count_dif));
        print('   total great great points = '+str(eq_count_great_great));
        print('   total great less points = '+str(eq_count_great_less));
        print('   total less great points = '+str(eq_count_less_great));
        print('   total less less points = '+str(eq_count_less_less));
    return False;
look_for_points(10,all=True, count = True)
\end{verbatim}
}